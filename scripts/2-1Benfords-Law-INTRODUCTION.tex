\chapter{Benford's Law Background and Applicability}

Benford's Law describes a particular behaviour of digits in a number. The lowest digits (such as 1 or 2) are most likely to occur in the first and second positions within a number (labelled as significant digits), their distribution approaches logarithmic distribution, its depiction of the law is shown latter in figure \ref{fig:FL} and the approximate relative frequencies described by formula \ref{BZ-general_first}. \cite{Cerqueti2202,Hronova2023,Newcomb1881}



While this is very useful for validation and fraud detection in many areas, it's theoretical validity still has not yet been proven even though there has been some explanations put forward. %\cite{Hronova2023,Cerqueti2202} introduction 
It has been demonstrated that this behaviour is common for numbers generated by various methods, including random linear combinations, aggregation of different datasets or random selection from such, and processes arising from multiplication (such as geometric series or exponential growth). Such numbers can be found in all fields of science, including census data, stock prices or the number of seconds between earthquakes. \cite{Hronova2023,kossovsky2014benford, Cerqueti2202} %section 1-15 

Benford's law also assists in understanding the discrepancy between the perception of randomness in numbers and the actuality. Many individuals hold the notion that numbers are distributed evenly across numerous datasets, or that repetition is minimal. This is exemplified by students being required to write out random numbers as part of an experiment - without prior knowledge of this law, it is very hard to write out random numbers that are actually random. Or accountants rounding off numbers as a part of their computations. \cite{kossovsky2014benford} % section 25, strana 8é, druhá polovina 
%This is also why various manual manipulations of numbers can be identified with high accuracy by BL.

The compliance of our data to the law can be expected when only minimal conditions are met. Then the conformity can be tested. This makes the law good for detecting presence of any fraud or various manual manipulations of numbers with high accuracy. \cite{kossovsky2014benford, Cerqueti2202,kossovsky2014benford} 

\section{History}

This section will describe the historical development of this phenomenon, demonstrating that in fact Frank Benford was not the first to identify this principle. And then we shall describe some of its initial applications. \cite{kossovsky2014benford,Hronova2023}

\subsection{Simon Newcomb (1835-1909)}

The notable Canadian-American astronomer and mathematician lived in the second half of the 19th century. And although he was recognized for his work in astronomy and physics, few people have noticed his article \emph{\uv{Note on the Frequency of Use of the Different Digits in Natural Numbers}}, where he had described the regularity in the occurrence of the first digits of a number, later known as Benford's law. He demonstrates this phenomenon by calculating the relative frequencies of occurrence of the first and second digits as noted in equations \ref{BZ-general_first} and \ref{BZ-general_second}. The behaviour of the remaining digits was only described by words not by equations. \cite{kossovsky2014benford, Newcomb1881, Hronova2023}  

\subsection{Frank Benford (1883-1948)}

Frank Benford, after whom the law was named, lived at the turn of the 19th and 20th centuries, he was an American mathematician, physicist and electrical engineer, but he also held up to 20 patents in the field of optical devices. He was unaware of Newcomb's earlier paper on the frequencies of natural numbers when he published his much longer paper \emph{\uv{The Law of Anomalous Numbers}}. \cite{kossovsky2014benford, Hronova2023}

In contrast to Newcomb, Benford sought to further test the validity of the rule. Thus, he compared 20 large datasets of different data types and systematically recorded the results of these tests. However, it was not without flaws. For instance, it lacked an accurate mathematical description of the phenomenon, and some of his assumptions regarding the origin of the numbers were also unsubstantiated. \cite{kossovsky2014benford, Hronova2023}

His work was preceded by the famous story of the logarithmic tables - he saw that the tables had worn pages only at the beginning, but were as good as new at the end, as saw Newcomb earlier, which made them both think about this phenomenon. Probably, those who used them most often needed to know the value of the lower numbers in their calculations more often than of the higher values.  \cite{Hronova2023}

\subsection{Theodore Preston Hill}

It is also worth mentioning Theodore Preston Hill, who, relying on Benford's work, proved the theory of mixed distributions in 1995 and added the necessary mathematical proofs. \cite{kossovsky2014benford} % section 25, strana 80, konec  {\color{blue}(zdroj? dočíst!)} 

\subsection{The first uses of BL in practice}  

In 1972, Hal Varian, then a master's student at the University of California at Berkley, came up with the idea of using Benford's law to detect faulty data in economics. This is the first documented use of BL. \cite{kossovsky2014benford} %section 25, strana 79

He worked for a real estate agency where he found a bug in a program that was used for running simulations. So he started thinking about detecting faulty data - how to differentiate data that is \emph{genuine} from flawed data in general. He had learned about BL. After a successful application, he wrote that Benford's Law was almost to the point of being \emph{\uv{numerological}}, since it was not as strongly mathematically supported at the time as it is today. He drew on practical results that suggested that the application was valid. \cite{kossovsky2014benford} %section 25

The first person to use the BL in a scientific paper was Charles Carslaw of the University of Canterbury in New Zealand in 1988. It was applied to some data from finance and accounting. He looked at how the earnings figures for New Zealand companies came close to the theoretical proportions of BL. It showed how, contrary to the law of the last digits, %\textcolor{blue}{(doplnit presnou zkratku co budu pouzivat)}
earnings and income are often rounded to multiples of powers of 10 - as well as the losses and expenses. \cite{kossovsky2014benford} %section 25, strana 80 

The 1990s then saw the addition of many more publications %notably by Mark Nigrini, C. W. Christian, S. Gupta and many others, who
introducing the first forensic methods of how to use BL to detect manipulation or fraud. \cite{kossovsky2014benford} %section 25, strana 80 

Since then this law is nowadays used as a standard test for many tax departments worldwide. Usually in the form of a routine test where the first digits are compared to see how they are distributed across the entire dataset. Often, it is discovered that the digits are distributed evenly if the data is artificially created or has been poorly manipulated. It is also used in accounting, auditing and other financially oriented industries as well as in other fields with risk of data fraud. \cite{kossovsky2014benford} % section 26, strana 81 

\section{Applicability of Benford's Law}

The law does not put many conditions on the data that will be analysed, yet there are some that must be met for reliable results. The following will be discussed in this section. 

\subsection{Correct usage}

When using Benford's Law to detect intentional fraud or data manipulation, the complete dataset is used, rather than a subset. Additionally the dataset must satisfy conditions other than those mentioned in the previous chapters. The dataset should be large enough - for datasets with less than 100 observations, BL analysis is not recommended. %\cite{kossovsky2014benford} % section 26, strana 82
Only non-negative data should be used for an analysis. For use in auditing, it is also recommended to eliminate small numbers, values less than, for example, 50, but that the portion removed be less than $10\%$. In addition, values that represent totals, summations, etc. should not be included if came from the same sample. \cite{kossovsky2014benford} % section 27, strana 88 

When detecting similar errors, it is often not enough to look at the distribution of the first digits only. It is advised to look at the last two digits to detect rounding. For a comprehensive analysis, \citeauthor{kossovsky2014benford} recommends a nine-step procedure:

\begin{figure}[h]
    \centering
    \label{fig:enter-label}
\begin{enumerate}
    \item First-digits distribution
    \item Second-digits distribution
    \item Combination of the first-two-digits distributions
    \item Combination of the first-three-digits distributions
    \item Combination of the last-two-digits distributions 
    \item Examination of first-order digital development
    \item Examination of second-order digital development
    \item Value repetition test
    \item Summation test   
\end{enumerate}
    \source{ \citeauthor{kossovsky2014benford}, \citeyear{kossovsky2014benford}}
\end{figure}

\subsection{Varying results from the expected distribution}

It is advised to generally focus on deviations from the theoretical distribution according to a given BL variation, specifically on the significantly higher values for the frequency of occurrence for a given digit. In graphical representation, these can be represented as spikes. There is minimal interest in the reduced frequencies, as any suspicious activity is likely to manifest itself through increased frequencies - when counting frequencies, an increase in one digit is much more detectable, with all other frequencies dropping slightly. \cite{kossovsky2014benford} % section 26, strana 85 

\begin{koment}
    grafický příklad jak vypadá hrot jakožto odchylka od konformity s BZ - pokud nebudu mít nic svého, Kossovsky strana 83. Potom i další ukázky podivných situací.
\end{koment}

%There will always be variation, because few datasets are $100\%$ \textit{Benfordian}.  \cite{kossovsky2014benford} % section 26, strana 84
Deviance is to be expected - but where should the line be drawn?  \citeauthor{kossovsky2014benford} suggests to distinguish between two terms: compliance and comparison. When researching the data's compliance to BL, the data is expected to follow the logarithmic distribution very closely, the focus in on detecting manipulation and if there is, if it is random or structural. For comparison the conditions are much loose. We don't assume any \textbf{prior population type (logarithmic distribution)}. For this our use, we will be assuming the data should follow the distribution and therefore observe data compliance.


% \begin{koment}
%     Pozor na stanovení N při testování pomocí chisq testu.
%     Je tam důležité odlišit, co je skutečně ta reálná populace - uvádí na příkladu, že populace mohou být sales revenue za celé čtvrtletí (57 tisíc záznamů) - unique price list, unique products... ale třeba sales revenue z malého coffee shopu bude třeba k nějaké populaci přihodit - je zde možné testovat compliance a u toho velkého vzorku ne? Nevím, jestli to chápu dobře... 

%     Pokud ten dataset existuje sám o sobě in its unique fashion, je to potom comparison, ne compliance. Compliance to bude když testujeme (za použití statistického testu) nějaký náhodný výběr. (takže třeba hlasy jen pro jednoho z kandidátů?) 

%     Pak se dal ukazují Z testy (pro jednotlivá pozorování) a pak ChiSq test pro cele. Nakonec je pak jeste zajimava smerodnatna odchylka jako measure odlisnosti. 
% \end{koment}

Next the deviations should not show signs of any systematic error or pattern. Some pattern may emerge by rounding numbers for example. The practice of rounding data is prevalent in the context of accounting, where values ending in 00 or 50 are frequently observed. This rounding process, which is often employed to streamline computations, can give rise to manipulation of higher orders, while being harmless in context. The prevalence of these values in practice is substantiated by empirical evidence. %\cite{kossovsky2014benford} %section 27, strana 92
However, such manipulations are unacceptable in, for example, electoral statistics, where votes simply cannot be rounded.
\cite{kossovsky2014benford}

\begin{koment}
    * insert příklad z analýzy volebních výsledků v Rusku, kde se zaokrouhlovalo, nebo odkaz na vyssi kapitolu kde se k tomu snad dostanu * 
\end{koment}

There are also numbers that someone must have made up at some point, in accounting, donations are a good example. That is a number that someone has determined. Just as, for example, prizes. Such numbers does not follow BL and will show high degrees of deviations. \cite{kossovsky2014benford}

Alternatively, the validity of BL can be influenced by changed external situations in regards to the data itself. In instance, how does sales data follow BL distribution when they have been affected by the pandemic in 2020? This has been answered by \citeauthor{Hronova2023} in \citeyear{Hronova2023}. Changes like this should be taken account and analysis should be adapted accordingly.


