\chapter{Benford's Law Background and Applicability}

%Benfordův zákon nám říká, že se v číslice v číslech chovají daným způsobem - na první a druhé pozici se s nejvyšší pravděpodobností vyskytují nejnižší číslice, a ty nejvyšší se naopak vyskytují s pravděpodobností nejnižší. Tyto proporce pak odpovídají logaritmickému rozdělení \cite{kossovsky2014benford}. 
%Benford's Law describes the behaviour of digits in a particular way. 

Benford's Law describes a particular behaviour of digits in a number. The lowest digits are most likely to occur in the first and second positions within a number, while the highest digits are most likely to occur in the lowest positions within a number. The proportions in question correspond to a logarithmic distribution. While this is very useful for validation and fraud detection in many areas, it's theoretical validity still has not yet been proven. %\cite{Hronova2023} introduction 
It has been demonstrated that this behaviour is common for numbers numbers generated by various methods, including random linear combinations, aggregation of distinct datasets or random selection from such, and processes arising from multiplication (geometric series or exponential growth). Such numbers can be found in all fields of science, including census data, stock prices or the number of seconds between earthquakes. \cite{Hronova2023} \cite{kossovsky2014benford} %section 1-15 



%Benfordův zákon také pomáhá pochopit, jak se liší lidské představy o náhodnosti čísel a jak je tomu ve skutečnosti. Představy mnohých lidí jsou, že se čísla vyskytují v mnoha datasetech rovnoměrně nebo že se málo či vůbec neopakují - účetní zaokrouhlí číslo, studenti, když mají zpaměti vypisovat náhodná čísla v rámci experimentu... A také proto se dají různé manuální manipulace s čísly poměrně dobře pomocí BL odhalit. 

Benford's law also assists in understanding the discrepancy between the perception of randomness in numbers and the actuality. Many individuals hold the notion that numbers are distributed evenly across numerous datasets, or that repetition is minimal. This is exemplified by accountants rounding off a number or students being required to write out random numbers as part of an experiment. This is also why various manual manipulations of numbers can be identified with high accuracy by BL.\cite{kossovsky2014benford} % section 25, strana 8é, druhá polovina


\begin{koment}
Comment on the conditions of what the dataset has to meet! 
\end{koment}

%Aby BZ platil, musí být čísla - soubor čísel, takzvaně \emph{benfordovský} - dostatečně velký a splňovat podmínku \ref{BZ-podminka} \cite{kossovsky2014benford}. %section 10

% \begin{equation}
%     \label{BZ-podminka}
%     \log(\text{90. percentil}) - \log(\text{10. percentil}) \ge 3
% \end{equation}

%\begin{koment}
% detailnější teoretická vysvětlení najdeme v sekcích 16-21 \cite{kossovsky2014benford}
% \end{koment}

%Bylo zjištěno, že takto se chovají čísla, která vznikla: jako náhodné lineární kombinace, agregací dostatečně odlišných datasetů, náhodně vybraná čísla z odlišných datasetů a procesy vzniklé násobením (například geometrické řady či exponenciální růst). \cite{kossovsky2014benford} %section 15 



\begin{koment}
    exponenciální růst - rychlý x pomalý se vlastně liší jen v časových jednotkách, když dostatečně zvětšíme interval mezi měřeními bude i ten nejpomalejší růst velmi rychlý. \cite{kossovsky2014benford}
\end{koment} %strana 64

%Taková čísla najdeme ve všech oblastech vědy. Ať už jde o data ze sčítání, adresy domů ve městech, ceny akcií nebo počet vteřin mezi zemětřeseními.  



\begin{koment}
zajímavý příklad užitečnosti BL je v \cite{kossovsky2014benford} v části 11 - zemětřesení (časy mezi otřesy nejprve velmi hezky seděly do BZ, ale pak při kombinaci jevů najednou nesedělo do BZ)
\end{koment}


\section{History}

Benford's law is  now used as a standard test for many tax departments worldwide. It is also used in accounting, auditing and other financially oriented industries. Often in the form of a routine test where the first digits are compared to see how they are distributed across the entire dataset. Often, it is discovered that the digits are distributed evenly if the data is artificially created or has been poorly manipulated. \cite{kossovsky2014benford} % section 26, strana 81 

However, this was not always the case. This section will examine the historical development of this phenomenon, demonstrating that in fact Frank Benford was not the first to identify this principle. And then we shall describe some of its initial applications. 

\subsection{Simon Newcomb (1835-1909)}

The notable Canadian-American astronomer and mathematician lived in the second half of the 19th century. And although he was recognized for his work in astronomy and physics, few people have noticed his article \emph{\uv{Note on the Frequency of Use of the Different Digits in Natural Numbers}}, where he had described the regularity in the occurrence of the first digits of a number, later known as Benford's law. He demonstrates this phenomenon by calculating the relative frequencies of occurrence of the first and second digits. The behaviour of the remaining digits was only described verbally.  \cite{kossovsky2014benford} \cite{Newcomb1881} 


%Této práci předcházel právě ten slavný příběh o logaritmických tabulkách, které na prvních stranách měly ošoupané strany, kdežto ke konci byly jako nové, protože ti, kteří je používali, nejčastěji při výpočtech potřebovali znát právě hodnotu těch nižších čísel.  {\color{blue}(zdroj?)}

His work was preceded by the famous story of the logarithmic tables - he saw that the tables had worn pages at the beginning, but were as good as new at the end, because those who used them most often needed to know the value of the lower numbers in their calculations more often than of the higher values.  \cite{Hronova2023}

%Významný kanadsko-americký astronom a matematik žil v druhé polovině 19. století. A přestože za svou práci v oblasti astronomie a fyziky byl uznáván, málokdo si povšiml jeho článku  \emph{\uv{Note on the Frequency of Use of the Different Digits in Natural Numbers}}, kde velmi přesně popisuje zákonitost ve výskytech prvních číslic čísla, později známou jako Benfordův zákon \cite{kossovsky2014benford}. Tento jev demonstruje spočítanými relativními frekvencemi výskytu první a druhé číslice. Chování dalších číslic jen slovně popisuje \cite{Newcomb1881}.

\subsection{Frank Benford (1883-1948)}

%Frank Benford, po němž byl zákon pojmenován, žil na přelomu 19. a 20. století, byl to americký matematik, fyzik a elektrický inženýr, byl ale také držitelem až 20 patentů v oblasti optických zařízení. O dřívějším Newcombově článku o frekvencích přírodních čísel nevěděl, když vydal svůj o mnoho delší článek \emph{\uv{The Law of Anomalous Numbers}} \cite{kossovsky2014benford}. 

Frank Benford, after whom the law was named, lived at the turn of the 19th and 20th centuries, he was an American mathematician, physicist and electrical engineer, but he also held up to 20 patents in the field of optical devices. He was unaware of Newcomb's earlier paper on the frequencies of natural numbers when he published his much longer paper \emph{\uv{The Law of Anomalous Numbers}}. \cite{kossovsky2014benford}



%Oproti Newcombovi se Benford snažil dále platnost pravidla ověřit. Porovnával tak 20 velkých datových souborů o různých datových typech a systematicky zaznamenával výsledky těchto testů \cite{kossovsky2014benford}. Rozhodně ale nebyl bezchybný, chyběl například korektně popsaný jev z hlediska matematiky a některé jeho předpoklady o vzniku čísel taky platné nejsou \cite{kossovsky2014benford}. 

In contrast to Newcomb, Benford sought to further test the validity of the rule. Thus, he compared 20 large datasets of different data types and systematically recorded the results of these tests. However, it was not without flaws. For instance, it lacked an accurate mathematical description of the phenomenon, and some of his assumptions regarding the origin of the numbers were also unsubstantiated. \cite{kossovsky2014benford}

%Zájem o jeho práci v poslední době nepochybně stoupá, a tak se v poslední době s interpretacemi jeho díla takzvaně roztrhl pytel. {\color{blue}(zdroj?)}


\subsection{The first uses of BL in practice}  

%V roce 1972 Hal Varian, tehdy student magisterslého studia na University of California v Berkley, přišel s nápadem použít Benfordův zákon pro odhalení chybných dat v ekonomii. Toto je první dokumentované použití BZ. \cite{kossovsky2014benford} %section 25, strana 79

In 1972, Hal Varian, then a master's student at the University of California at Berkley, came up with the idea of using Benford's law to detect faulty data in economics. This is the first documented use of BL. \cite{kossovsky2014benford} %section 25, strana 79


%Pracoval pro realitní agenturu, kde odhalil chybu v programu, který vytvářel simulace. Dále proto přemýšlel o odhalování chybných dat - jak obecně odhalit data, která jsou \emph{přirozená} od těch chybných. Dozvěděl se o BZ. Čísla, se kterými pracoval, by se měla chovat podle BL - splňovala dříve zmíněné podmínky (being digitaly logarithmic). Po úspěšném použití napsal, že Benfordův zákon je skoro až numerologický, neboť v té době nebyl matematicky podložen tak silně, jako dnes. Vycházel z praktických výsledků, které nasvědčovaly, že použití sedí. \cite{kossovsky2014benford} %section 25

He worked for a real estate agency where he found a bug in a program that was used for running simulations. So he was also thinking about detecting faulty data - how to detect data that is \emph{genuine} from flawed data in general. He had learned about BL. After a successful application, he wrote that Benford's Law was almost to the point of being \emph{\uv{numerological}}, since it was not as strongly mathematically supported at the time as it is today. He drew on practical results that suggested that the application was valid. \cite{kossovsky2014benford} %section 25



%První osobou, která použila BZ ve vědeckém článku, byl Charles Carslaw z University of Canterbury na Novém Zélandu v roce 1988. Byl použit na data z financí a účetnictví. Sledoval, jak se číslice z výdělků v novozélandských firmách blíží teoretickým proporcím BZ. Ukázal, jak se oproti zákonu o druhé číslici, často výdělky/příjmy zaokrouhlují na násobky mocnin 10 - stejně tak jako ztráty a výdaje.  \cite{kossovsky2014benford} %section 25, strana 80 

%The first person to use the BL in a scientific paper was Charles Carslaw of the University of Canterbury in New Zealand in 1988. It was applied to data from finance and accounting. He looked at how the earnings figures for New Zealand companies approached the theoretical proportions of BL. It showed how, contrary to the law of the second digit, earnings/income are often rounded to multiples of powers of 10 - as well as losses and expenses. \cite{kossovsky2014benford} %section 25, strana 80 

The first person to use the BL in a scientific paper was Charles Carslaw of the University of Canterbury in New Zealand in 1988. It was applied to some data from finance and accounting. He looked at how the earnings figures for New Zealand companies came close to the theoretical proportions of BL. It showed how, contrary to the law of the last digits \textcolor{blue}{(doplnit presnou zkratku co budu pouzivat)}, earnings/income are often rounded to multiples of powers of 10 - as well as the losses and expenses. \cite{kossovsky2014benford} %section 25, strana 80 

%V devadesátých letech pak přibyly další mnohé publikace, zejména od  Marka Nigrini, C.W Christiana, S. Gupty a mnohých dalších, kteří představovaly první forenzní metody jak BZ použít při odhalování manipulací či nepravostí. 

The 1990s then saw the addition of many more publications, notably by Mark Nigrini, C. W. Christian, S. Gupta and many others, who introduced the first forensic methods of how to use BL to detect manipulation or fraud. \cite{kossovsky2014benford} %section 25, strana 80 

 \subsection{Theodore Preston Hill}

%Za zmínku také stojí Theodore Preston Hill, který v závislosti na Benfordovu práci dokazál teorii o smíšených rozděleních v roce 1995 a doplnil potřebné matematické důkazy {\color{blue}(zdroj? dočíst!)} + \cite{kossovsky2014benford} % section 25, strana 8é, konec 

It is also worth mentioning Theodore Preston Hill, who, relying on Benford's work, proved the theory of mixed distributions in 1995 and added the necessary mathematical proofs. {\color{blue}(zdroj? dočíst!)} + \cite{kossovsky2014benford} % section 25, strana 80, konec  

%\subsection{In the present day}

%V současnosti se již používá jako standardní test pro mnohá daňová oddělení celosvětově. Dále se pak setkáváme s použitím v účetnictví, auditu a dalších finančně orientovaných odvětvích. Často formou rutinního testu, kdy se porovnává, jak jsou první číslice rozděleny napříč celého datasetu. Často se stane, že se odhalí, že číslice jsou rozděleny rovnoměrně, jedná-li se o uměle vytvořená data nebo data, která byla neumě upravená. \cite{kossovsky2014benford} % section 26, strana 81 



\section{Correct usage}

%Při používání Benfordova zákona pro odhalování úmyslného podvodu či manipulace s daty se používá dataset celý, nikoli výběr z něj. Dále musí dataset splňovat i další podmínky, než ty, které byly zmíněny v předchozích kapitolách \textcolor{blue}{\emph{doplnit odkaz na konkrétní kapitolu, sjednotit možná ty podmínky na jedno místo a pak napsat -as mentioned before-}}. Dataset by měl být dostatečně velký - pro soubory dat s méně než 100 pozorováními se analýza pomocí BZ nedoporučuje. \cite{kossovsky2014benford} % section 26, strana 82
%K analýze by se měla používat pouze data s nezápornou hodnotou. Pro použití v auditingu se doporučuje eliminovat také malá čísla, hodnoty nižší například než 50, ale aby odstraněná část byla menší než $10\%$. Dále by se neměly zahrnovat hodnoty, které představují součty, souhrny a podobně, pokud pochází z té jedné a té samé firmy (vztahují se k té jedné a té samé věci). Týká-li se to stejné věci čísla se budou více duplikovat. Nevadí to pak u více firem a odvětví průmyslu, kde těch dat bude více, proto to nebude dělat neplechu. \cite{kossovsky2014benford} % section 27, strana 88 

When using Benford's Law to detect intentional fraud or data manipulation, the complete dataset is used, rather than a subset. Additionaly the dataset must satisfy conditions other than those mentioned in the previous chapters. The dataset should be large enough - for datasets with less than 100 observations, BL analysis is not recommended. \cite{kossovsky2014benford} % section 26, strana 82
Only non-negative data should be used for analysis. For use in auditing, it is also recommended to eliminate small numbers, values less than, for example, 50, but that the portion removed be less than $10\%$. In addition, values that represent totals, summations, etc. should not be included if they come from the same company (refer to the same thing). \cite{kossovsky2014benford} % section 27, strana 88 



%(dá se tak odhalit zaokrouhlování), při odhalení podivností při pohledu na první číslici se hodí přidat jednu až dvě další následné číslice pro zúžení potencionálních \emph{\uv{chybných kousků}}. \cite{kossovsky2014benford} 

%Při odhalování podobných chyb často nestačí pohlížet na rozdělení pouze prvních číslic. Dále se díváme také na poslední dvě číslice, čímž se dá odhalit například zaokrouhlování. Pro ucelenou analýzu Kossovsky doporučuje postup o devíti krocích \cite{kossovsky2014benford}: % section 26, strana 82

When detecting similar errors, it is often not enough to look at the distribution of the first digits only. We also look at the last two digits to detect rounding, for example. For a comprehensive analysis, \citeauthor{kossovsky2014benford} recommends a nine-step procedure
\cite{kossovsky2014benford}: % section 26, strana 82
\textcolor{blue}{\emph{jak to správně ocitovat?}} 

\begin{enumerate}
    \item rozdělení prvních číslic
    \item rozdělení druhých číslic 
    \item rozdělení pro kombinace prvních dvou číslic 
    \item rozdělení pro kombinace prvních tří číslic
    \item rozdělení pro poslední dvě číslice 
    \item rozdělení prvních číslic, rozděleno na intervaly po řádech \\ \textcolor{blue}{\emph{intervaly (0.1, 1), (1, 10)... a tak}}
    \item rozdělení druhých číslic, rozděleno na intervaly po řádech 
    \item test o opakování čísel 
    \item summation test \textcolor{blue}{???}    
\end{enumerate}

\begin{koment}
    toto rovnou vlozit jako prilohu z knihy, cely to ocitovat - mozna jako tabulku? 
\end{koment}

%Zaměřujeme se na odchylky od teoretického rozdělení podle dané variace BZ v podobě výrazně vyšších hodnot u četnosti výskytu pro danou číslici. Graficky to představují tzv. \emph{hroty}. Snížené četnosti nás tolik nezajímají - jednoduše proto, že pokud se nějaká podezřelá aktivita má projevit, je to právě přes zvýšené četnosti. Když počítáme četnosti, přírůstek jednoho znaku je mnohem lépe detekovatelnější - všechny ostatní četnosti trochu poklesnou. \cite{kossovsky2014benford} % section 26, strana 85

We focus on deviations from the theoretical distribution according to a given BL variation, specifically on the significantly higher values for the frequency of occurrence for a given digit. In graphical representation, these are represented by the spikes. There is minimal interest in the reduced frequencies, as any suspicious activity is likely to manifest itself through increased frequencies. When counting frequencies, an increase in one trait is much more detectable, with all other frequencies dropping slightly.\cite{kossovsky2014benford} % section 26, strana 85 


\begin{koment}
    grafický příklad jak vypadá hrot jakožto odchylka od konformity s BZ - pokud nebudu mít nic svého, Kossovsky strana 83. Potom i další ukázky podivných situací.
\end{koment}

There will always be variation, because few datasets are $100\%$ Benfordian. Still, the deviations should not show signs of any systematic error or pattern. \cite{kossovsky2014benford} % section 26, strana 84

%Odchylky se budou projevovat vždycky, protože málokterý soubor je $100\%$ benfordovský. Přesto ale platí, že odchylky by neměly vykazovat známky nějaké systematické chyby či vzoru. \cite{kossovsky2014benford} % section 26, strana 84

\begin{koment}
    Dala by se tady udělat nějaká reziduální analýza? Scatterploty či Shapiro-Wilk test jako u regrese?  
\end{koment}


\subsection{Harmless manipulation} 

%Záleží na odvětví a kontextu dat. V datech z účetnictví se často zaokrouhluje - proto tam budou hodnoty končící na 00 nebo 50 mnohem častější. Takhle vzniklé manipulace ovlivní hlavně rozdělení vyšších řádů. Nemusí to být ale jen zaokrouhlováním. Tyto hodnoty prostě můžou být skutečně častější, jak je známé z praxe. \cite{kossovsky2014benford} %section 27, strana 92
The practice of rounding data is prevalent in the context of accounting, where values ending in 00 or 50 are frequently observed. This rounding process, which is often employed to streamline computations, can give rise to manipulation of higher orders. The prevalence of these values in practice is substantiated by empirical evidence. \cite{kossovsky2014benford} %section 27, strana 92
However, such manipulations (e.g. rounding) are unacceptable in, for example, electoral statistics, where votes simply cannot be rounded.

%Takovéto manipulace (např. zaokrouhlování) jsou ale nepřijatelné třeba ve volební statistice, kde se hlasy prostě zaokrouhlit nedají.   

\begin{koment}
    * insert příklad z analýzy volebních výsledků v Rusku, kde se zaokrouhlovalo, nebo odkaz na vyssi kapitolu kde se k tomu snad dostanu * 
\end{koment}

%Dále jsou čísla, která prostě někdo někdy musel vymyslet - v účetnictví jsou dobrým příkladem třeba dary. To je číslo, které někdo určil. Stejně, jako třeba ceny. \textcolor{blue}{\emph{psychologie cen a tvorba cen}} 

There are also numbers that someone must have made up at some point - in accounting, donations are a good example. That is a number that someone has determined. Just as, for example, prizes. \textcolor{blue}{\emph{psychologie cen a tvorba cen}} 

\begin{koment}
    tady bych se měla porozhlídnout po zdrojích, co se nezabývají hlavně auditem a účetnictvím. 
\end{koment}




%\subsection{Kritika BZ a kdy by se neměl používat} 

% \begin{koment}
%     tady v této části bych měla popsat co psali v tom článku od Cambridge? \textbf{Benford’s Law and the Detection of Election Fraud} toto 
% \end{koment}

\subsubsection{Validity in varying conditions} 

The validity of BL can be influenced by changed external situations in regards to the data itself. In instance, how does sales data follow BL distribution when they have been affected by the pandemic in 2020? This has been answered by \citeauthor{Hronova2023}. 


Changes like this should be taken account and analysis should be adapted accordingly. 


