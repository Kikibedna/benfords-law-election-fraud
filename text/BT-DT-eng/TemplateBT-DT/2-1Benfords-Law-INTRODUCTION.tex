\chapter{Benford’s Law and Election Fraud} % and Applicability

Benford's Law describes a particular behaviour of digits in a number. The lowest digits (such as 1 or 2) are most likely to occur in the first and second positions within a number. It has been demonstrated that this behaviour is common for numbers generated by various methods, including random linear combinations, aggregation of different datasets or random selection from such datasets, and processes arising from multiplication (such as geometric series or exponential growth). Such numbers can be found in all fields of science, including census data, stock prices, or the time intervals between earthquakes. The historical context will be explained in this chapter. \cite{Hronova2023,kossovsky2014benford, Cerqueti2202} %section 1-15 

While phenomena like this are very useful for validation and fraud detection in many areas, it also helps to understand the discrepancy between the perceived and actual randomness in numbers. Many individuals believe that numbers are distributed evenly across numerous datasets or that repetition is minimal. This is exemplified by students being required to write out random numbers as part of an experiment, without prior knowledge of this law, people struggle to generate truly random numbers manually. \cite{kossovsky2014benford, Beber2012} % section 25, strana 80, druhá polovina 

This is also why various manual manipulations of numbers can be identified with high accuracy by BL. In an election context, the law is suited for detecting manual data manipulation. For us to analyse election data, it is necessary to understand where and how the manipulation can happen. This will be described in the second section of this chapter. 

\section{History of Benford's Law}

Benford's Law has been rediscovered multiple times throughout history. In fact, Frank Benford was not the first to identify this principle, yet still, it holds his name. This section will describe who was the first to write about it, enlighten its historical development and comment on some of its initial applications.

%Then, some of its initial applications will be described. %\cite{kossovsky2014benford,Hronova2023}

\subsection{Simon Newcomb (1835-1909)}

This notable Canadian-American astronomer and mathematician lived in the second half of the 19th century. And although he was recognized for his work in astronomy and physics, few people have noticed his article \textit{\citetitle{Newcomb1881}}, where he had described the regularity in the occurrence of the first digits of a number, later known as Benford's law. He demonstrates this phenomenon by calculating the relative frequencies of occurrence of the first and second digits as noted in equations \ref{BZ-general_first} and \ref{BZ-general_second}. The behaviour of the remaining digits was only described by words, not by equations. \cite{kossovsky2014benford, Newcomb1881, Hronova2023, SimonNewcomb}  

\subsection{Frank Benford (1883-1948)}

Frank Benford, after whom the law was named, lived at the turn of the 19th and 20th centuries. He was an American mathematician, physicist and electrical engineer, and held up to 20 patents in the field of optical devices. He was unaware of Newcomb's earlier paper on the frequencies of natural numbers when he published his much longer paper \textit{\citetitle{Benford1938}}. In contrast to Newcomb, Benford sought to further test the validity of the rule. Thus, he compared 20 large datasets of different data types and systematically recorded the results of these tests. However, his approach had flaws. For instance, it lacked an accurate mathematical description of the phenomenon, and some of his assumptions regarding the origin of the numbers were also unsubstantiated. His work was preceded by the famous story of the logarithmic tables - he saw that the tables had worn pages only at the beginning, but were as good as new at the end, as Newcomb had observed earlier, which made them both think about this phenomenon. Probably, those who used them most often needed to know the value of the lower numbers in their calculations more often than the higher values.  \cite{Benford1938, kossovsky2014benford, Hronova2023}

\subsection{The First Uses of Benford's Law in Practice}  

In 1972, Hal Varian, then a master's student at the University of California at Berkeley, came up with the idea of using Benford's law to detect faulty data in economics. This is the first documented use of BL. %\cite{kossovsky2014benford} %section 25, strana 79
He discovered a bug in a real estate agency’s simulation program. So he started thinking about detecting faulty data - how to differentiate data that is \emph{genuine} from flawed data in general. Previously, he had learned about BL. After applying the law successfully, he described it as \emph{\uv{almost numerological}}, since it was not as strongly mathematically justified at the time as it is today. He drew on practical results that suggested that the application was valid. \cite{kossovsky2014benford} %section 25

The first person to use the BL in a scientific paper was Charles Carslaw of the University of Canterbury in New Zealand in 1988. It was applied to some data from finance and accounting. He looked at how the earnings figures for New Zealand companies came close to the theoretical proportions of BL. It showed how, contrary to the BL for the last digits, %\textcolor{blue}{(doplnit presnou zkratku co budu pouzivat)}
earnings and income are often rounded to multiples of powers of 10, as well as the losses and expenses. \cite{kossovsky2014benford} %section 25, strana 80 

The 1990s then saw the addition of many more publications %notably by Mark Nigrini, C. W. Christian, S. Gupta and many others, who
introduced the first forensic methods of how to use BL to detect manipulation or fraud. It is also worth mentioning Theodore Preston Hill, who, relying on Benford's work, proved the theory of mixed distributions in 1995 and added the necessary mathematical proofs. %\cite{kossovsky2014benford} % section 25, strana 80, konec  {\color{blue}(zdroj? dočíst!)} 
Since then, this law is nowadays used as a standard test for many tax departments worldwide. Usually in the form of a routine test, where the first digits are compared to see how they are distributed across the entire dataset. Often, it is discovered that the digits are distributed evenly if the data is artificially created or has been poorly manipulated. It is also used in accounting, auditing and other financially oriented industries as well as in other fields with a risk of data fraud. \cite{kossovsky2014benford} % section 26, strana 81 


\section{Use of Benford's Law in election context}

% \begin{koment}
%     How do elections differ from other use cases for BL? 
% \end{koment}



\begin{quote}
\uv{\textit{Election results closely conform to Benford’s Law. This is so since electoral
results are simply manipulated fractional population data; as in 31\% of population
voting for candidate A promising policy bundle X, 27\% voting for candidate B
promising policy bundle Y, and the remaining 42\% of the population simply not
voting in order to preserve their human dignity and freedom by not participating
in such a ridiculous charade, patiently awaiting the triumphant arrival of direct and
participatory democracy instead. Since population data itself conforms strongly to
the law, by the scale invariance principle, the same conformity should be found
here for fractional population values. Since population data itself conforms strongly to
the law, by the scale invariance principle, the same conformity should be found
here for fractional population values. A more careful examination of the underlying statistical process reveals that this is ultimately a Random Linear Combination from random logarithmic data. This confers election results an even more logarithmic aura than mere population data}}
\end{quote}

\begin{flushright}
  bude přepsáno, zdrojem je opět \citeauthor{kossovsky2014benford} 
\end{flushright}





In digit-based diagnostics, the literature has recommended focusing more on the last digits for analysing manual counts, eg. election results. % \cite{Beber2012, Deckert2011} %, rather than on the first digits
Authors \citeauthor{Beber2012} recommend testing these biases in number generation according to psychology: last digits should occur with equal frequency, and there should be some repetition among the numbers and digits. Note that these biases apply when fraud is done manually, and the fraud comes from a number generation done by a human. 

On the other hand \citeauthor{Deckert2011} suggests to focus on last and next to last digits, since the fraud is expected to occur by rounding off the counts to tens and fives. While he does not recommend using BL and the second-digit test for detecting election fraud, but as \citeauthor{Beber2012} in \citeyear{Beber2012} described, BL is a suitable tool for detecting manual record manipulation. Further, \citeauthor{Mebane2011} has disputed the arguments of \citeauthor{Deckert2011} in \citeyear{Mebane2011}. BL and the second-digit test can be useful if applied correctly and with proper theoretical grounding, more on that later.

Additionally, Benford's Law should not be used a as stand-alone tool and should be accompanied by other relevant tests. \citeauthor{Lebeda2021} in paper \citetitle{Lebeda2021} have analysed election from a quantitative as well as a qualitative perspective by analysing turnout variance using regression models on an election office level and by interviewing members of election committees. Analysing election fraud is a complex process and this paper is focusing only on one part, detecting abnormalities in election counts. 


In the following lines, election process and vote counting will be described. Especially with a focus on presidential election specifics. Furthermore election integirity and why it matters will be unraveled. By the end of this chapter the common types of election fraud will be categorized.  



\subsection{Elections} 

An election is a formal decision-making process during which the given population selects a representative or a group of representatives to hold public office or decide on a political proposition by voting. 
Usually, elections are held in one day announced in advance. The goal is to provide everyone with an equal opportunity to participate. Voters go to polling stations, where their identity is checked and they can perform the vote. Performance of the vote can change depending on the region and country. \cite{election} 


There are four principles of election, that are essential in democratic societies: 

\begin{koment}
    \begin{itemize}
        \item vseobecnost 
        \item rovnost
        \item tajné právo 
        \item přímé právo 
    \end{itemize}
\end{koment}

\subsubsection*{Elections in Czechia}

Election takes place on Friday from 2 pm till 10 pm and Saturday from 8 am till 2 pm. Voters put their votes into a ballot box in a designated envelope, stamped with an official stamp and in designated colour. Votes, that are not in the envelope, are not valid. The envelope has also a second functionally, besides the validity of the vote, it is also providing voter by keeping their vote a secret. That is another important principle election must match. The counting of the votes start right when the voting office is closed. It has to be locked and only members of voting committee must be present. \cite{volby}


\subsubsection*{Gathering and analysing results}

% \begin{koment}
% Volební okrsky jsou koncipovány tak, aby byly cca po 1000 lidech. 

% https://mv.gov.cz/volby/clanek/volebni-okrsky.aspx (zveřejněno dne 19. července 2021, převzato dne 7.2.2025) 

% https://csu.gov.cz/rso/volebni-kraje-volebni-obvody-a-volebni-okrsky 



% Možný zdroj pro data: 

% https://www.nationalelectionsdatabase.com/ 


% Mozna nejaka reflexe vlastni zkusenosti jako clenky volebni komise? pripadne do uvodu jako motivace k praci
% \end{koment}


% \subsubsection*{Presidential election specifics}

% jestli nějaké jsou 

\subsection{Electoral Integrity and Election Fraud}

\begin{koment}
    %https://acpo.vedeckecasopisy.cz/publicFiles/002370.pdf


Volební integrita je sobour mezinárodních a globálních norem, vztahují se k celému volebnímu procesu - období před volbami, při nich a potom (sčítání a vyhodnocování volebních výsledků). Během celého tohoto procesu mohou nastat některá pochybení, manipulace, a to jak úmyslné, legální i ne, tak i neúmyslné. Všechny připívající do integrity voleb jako takových. 

(Electoral Integrity) Transparentní volby jsou základním kamenem demokracie. Důvěra v legitimní a transparentní volby také souvisí s důvěrou společnosti v demokracii a demokratické principy, čehož s oblibou využívají populistická a autoritářská hnutí a strany. 

Proto by snaha neměla být zaměřena na diskreditaci již uplynulých voleb, ale implementace analytických nástrojů by mohla tyto pochyby dříve a lépe (efektivněji) rozptýlit.

Znevažování voleb v minulosti vedlo i nebezpečným situacím - např. neustálé znevažování voleb (označování za ukradené) na podzim v roce 2020 a mobilizace voličstva vedly k útokům na Kapitol v lednu poté. Podobné snahy mohou vést k postupné erozi demokracie a vzestupu populistů, autoritářů a dalších fancy lidí. 

Zdroj: \cite{Lebeda2021}, bude přepsáno

\end{koment}

\begin{quote}
    \uv{\textit{The integrity of the electoral process can be maintained by a variety of devices and practices, including a permanent and up-to-date register of voters and procedures designed to make the registration process as simple as possible. In most jurisdictions elections are held on a single day rather than on staggered days. Polling hours in all localities are generally the same, and opening and closing hours are fixed and announced, so that voters have an equal opportunity to participate. Polling stations are operated by presumably disinterested government officials or polling clerks under governmental supervision. Political party agents or party workers are given an opportunity to observe the polling process, which enables them to challenge irregularities and prevent abuses. Efforts are made to maintain order in polling stations, directly through police protection or indirectly through such practices as closing bars and liquor stores. The act of voting itself takes place in voting booths to protect privacy. Votes are counted and often recounted by tellers, who are watched by party workers to ensure an honest count. The transmission of voting results from local polling stations to central election headquarters is safeguarded and checked.}}
\end{quote}

  \begin{flushright}
  - as described in Encyclopedia Britannica, in \citeyear{election}. přepíšu
  \end{flushright}

% \begin{koment}
% https://denikn.cz/1694101/zelenskeho-tym-se-pripravuje-na-volby-mely-by-byt-driv-nez-vyprcha-efekt-ovalne-pracovny/

% aktuálnost situace - co kdyz budou volby zmanipulovane? muze vyhrat/prohrat valku 

% \end{koment}


Election fraud, manipulation or voter fraud and voter rigging are all names of electoral abuses. These are forms of dishonest practices aimed to change the outcome of elections, usually to undermine democratic processes. All of these come to be the exact opposites of electoral integrity. They can affect the legitimacy of electoral processes which can lead to less people participating in election or increase the distrust and decrease the support of democratic political process. Some of these practices can be legal, but morally unacceptable. That's why it is necessary for public institutions to make efforts to detect them and to protect democratic society, specifically public trust in free and fair democratic elections and their results. \cite{Lehoucq2003, Levitt2007, election, Lebeda2021}

Definition of election fraud is a bit complicated. \citeauthor{Levitt2007} mentioned that using the term \uv{\emph{voter fraud}} can lead to an impression that it is somehow the voter's fault. \citeauthor{Lehoucq2003} was focusing on systematic influences that may change the voter's decision, such as voter intimidation or bribery, which is not the focus of this thesis. That is why in this paper the authors have stuck with the term election fraud, where the focus is shifted towards the election system as a whole, from how the votes are being collected, counted, imputed into database and then analysed. 

Election fraud is nevertheless the act of purposeful \emph{(ovlivňování, manipulace... něco)} of election process usually  done in secret by manipulating the voters or the results itself. The topology of such has more opinions. In this paper, we will focus of just a handful named examples and in the analysis part, we are analysing only those, that can be diagnosed digitally. \cite{Lebeda2021}



% \begin{koment}
% Honest mistakes do happen, but how can we distinguish them from purposeful manipulation? 
% What makes a mistake into a effort to change election result is...
% \end{koment}

There are many examples of election fraud throughout history. \citeauthor{Lehoucq2003} made a comprehensive literature analysis on this topic. In history a more direct approach was more common - practices as buying votes or direct persuasion. Nowadays, the general direction has shifted towards manipulation of elections results. In the next lines there will be some common types of election fraud further explained. \cite{Levitt2007} 

\subsubsection*{Voter impersonation} 
Pretending to be someone else and voting as that person . This is very rare and usually can be explained as a simple error, eg. a typo, mistaking two people with matching birthdays, matching to dead people, etc. And as \citeauthor{Levitt2007} emphasizes, it is not a very effective strategy to win election. There is also voting as someone who is not eligible to be voting - a minor, an alien or a non-citizen. Similarly, the term \textbf{duplicate voting} describes a situation, where a person is voting as the same person, but on multiple locations. \cite{The_Heritage_Foundation_2024}

\subsubsection*{Voter caging} %hodne americkej context, uplne nevim jestli nejak nutny? 
The process of first mass mailing to registered voters with the goal of identifying those that are enrolled under invalid address, based on the return rate of send mail, and then debating voters' registration. Targeting the opposite party's voters. There are many reason as to why it has not been delivered, so the registration can be valid and this is not a good way to check the validity. \cite{Levitt2007}

\subsubsection*{Vote Buying} 
Agreement or schemes based on an exchange of votes by voters and political party or representatives for money or other forms of rewards. Reward can be anything, honorable mentions Babisovi, jak vozi duchodce do volebnich mistnosti. \cite{Lebeda2021, Levitt2007, The_Heritage_Foundation_2024}

\subsubsection*{Record manipulation}
The primary focus of this thesis - altering the vote count. Usually done in the phase after voting, while the results are being collected. \cite{Lebeda2021, The_Heritage_Foundation_2024}

% \textbf{bullet stuffing} - hromadné vkládání hlasů, přidávají se falešné hlasy v průběhu voleb 

% \textbf{ballot stuffing} - manipulace s počtem odevzdaných hlasů konkrétního kandidáta 

% \textbf{gerrymandering} - manipulace s volebními okrsky ve prospěch voleného kandidáta 

