\chapter*{Introduction}
\addcontentsline{toc}{chapter}{Introduction}


\begin{koment}
In the introduction of the thesis, the author explains why she/he choose the 
chosen topic, thus the \textbf{motivation} of the whole thesis. The introduction must not 
miss the precisely formulated \textbf{main goal} of the thesis (or sub-goals), the 
\textbf{methodology} of the whole thesis (or research questions of hypotheses) should be 
outlined. It is also common practice to outline the \textbf{main results/outcomes} of the 
thesis.

The introduction is followed by individual \textbf{numbered chapters} divided into 
subchapters.



Firstly, an overview of Benford’s Law and its relevance in detecting anomalies in datasets, with a specific focus on its application in electoral data, is provided. A discussion of historical case studies where Benford’s Law has been used will be presented as well.

Next, the statistical techniques and models that will be applied in the analysis of election data are introduced in detail. This includes a description of Benford's Law, its mathematical formulation, and its expected distribution in naturally occurring datasets. Other complementary statistical methods used to strengthen the analysis will also be explained.

Lastly, the methodology will be applied to real election data to assess the presence of anomalies. Detailed steps of the analysis process will be outlined, including data collection, preprocessing, and application of Benford's Law. The results will be discussed, and conclusions will be drawn based on the statistical findings.
\end{koment}




\subsection*{ukolnicek}

\begin{itemize}
    \item TODO generalizace BL 
    \item IN PROGRESS doplnit dalsi zdroje at to neni jen Kosovsky 
    \item IN PROGRESS dalsi metodologicky srandicky - pravdepodobnostni rozdeleni, chyby atd 
    \item PENDING ziskat data 
    \item PENDING vycistit a zanalyzovat data 
    \item TODO metodologie - dopsat az budu mit ucebnice od Marka a od Male 
    \item ??? Theodore Preston Hill zdroj - chci si precist to puvodni? 
    \item ??? Benfordovo dilo to samy 
\end{itemize}

