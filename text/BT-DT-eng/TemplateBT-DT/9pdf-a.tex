\chapter{PDF/A format}

Electronic form of final
work must be submitted in PDF/A format level 1a or 2u. They are
PDF profiles that determine which PDF properties are allowed to use
to make the documents suitable for long-term archiving and further automatic
processing. Next we will deal with level 2u, which we bet on \LaTeX{}.

The most important requirements of PDF/A-2u include:

\begin{itemize}

\item All fonts must be built into the document. They are not allowed
links to external fonts.

\item Fonts must contain a ToUnicode table that defines the conversion from encoding
characters used inside a Unicode font. This makes it possible from the document
reliably extract text.

\item The document must contain metadata in XMP format and, if colored,
then also the formal specification of color space.

\end{itemize}

This template uses the {\ tt pdfx} package, which \LaTeX{} can set up
to meet the requirements of PDF/A. Metadata in XMP is generated automatically by
information in the file {\ tt thesis.xmpdata} (you can refer to the generated file
see in {\ tt pdfa.xmpi}).

The correctness of PDF/A can be checked using an online validator: \url{https://www.pdf-online.com/osa/validate.aspx/}.

If the file is not valid, common causes include less use
common fonts (which are inserted only in bitmap format and/or without
unicode tables) and embedding images in PDF, which are standard in themselves
PDF/A do not meet.

This is likely to be the case for images created by many different programs.
In this case, you can try to convert the image to PDF/A using
GhostScript, for example, as follows:

\begin{verbatim}
        gs -q -dNOPAUSE -dBATCH
           -sDEVICE=pdfwrite -dPDFSETTINGS=/prepress
           -sOutputFile=vystup.pdf vstup.pdf
\end{verbatim}
