\chapter{Application on Election Data} %of Benford's Law 

V nasledujicím segmentu budeme analyzovat výsledky prezidentských voleb v Rusku, u kterých se předpokládá nějaký fraud, ve Spojených státech, kde můžeme být svědky \textbf{gerrymanderingu}, akorát to asi není topic této práce, když koukáme na fraud perspektivou digits... takže by tam žádný fraud být nejspíš neměl. Dále nahlédneme do Estonska, kde mají zdigitalizovaný volební systém, dá se tam volit i přes mobil, země, ke které vzhlížíme co se digitalizace týče. No a naši pouť zakončíme prezidentskými volbami v Česku, kde opět neočekáváme nějaký závratný fraud.  


\section{Russia}
https://www.theguardian.com/news/datablog/2012/mar/05/russia-putin-voter-fraud-statistics


\section{USA} 


\section{Czechia Presidential elections in 2023}
priority 

\begin{koment}
VERBATIM, PŘEPSAT:

V České republice se opakovaně setkáváme s poukazy na nejrůznější vady volebního procesu. Běžně se můžeme dočíst o kupčení s hlasy, o podezřele vysokých počtech zneplatněných hlasů, hromadných účelových změnách trvalého bydliště před konáním komunálních voleb, účelovém svážení občanů k volebním místnostem , nedostatečné kontrole hlasovacích místností6 zejména v noci z pátka na
sobotu, nekvalitní práci volebních komisí apod. \cite{Lebeda2021}

Reakce ze strany
státu na tyto podněty však byla dlouhou dobu spíše vlažná. Opatření, která by tyto ne-
dostatky eliminovala, byla v minulosti přijímána dosti pomalu. Transparentnosti bychom přitom měli věnovat velkou pozornost, protože spolu s důslednější kontrolou voleb totiž mohou vést k vyšší důvěře v demokratický režim, k vyšší volební účasti a tím pádem i k vyšší
legitimitě volených orgánů. \cite{Lebeda2021}

Podezřelé: 
Rozdíl mezi vydanými a odevzdanými obálkami, podíl neplatných hlasů (špatně odevzdané hlasy), 100 \% využitých preferenčních hlasů (celostátní průměr je asi 12 \%), 100 \% účast... Toto často vede na špatnou práci volební komise. \cite{Lebeda2021}

Dalším problémem je to, že se volební komise málo obměňují, tudíž k těmto chybám může docházet opakovaně a systematicky. \cite{Lebeda2021}

\end{koment}



\section{Estonia?}
non priority 


co mi vyšlo 
