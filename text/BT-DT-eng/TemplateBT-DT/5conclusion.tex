\chapter*{Conclusion}
\addcontentsline{toc}{chapter}{Conclusion}

In the conclusion, the author summarizes the individual conclusions, analyses 
and interpretations. It is useful if the author concludes by acknowledging the 
limits of his/her work and possibilities of continuing the topic.

\begin{koment}

    Vyznam BL pro aplikaci v digital fraud detection pro volby 

    Jednoduchost implementace 

    Ceske prezidentske volby: 
    \begin{itemize}
        \item transparentni uz jen tim, ze je k datum snadny pristup na CZSO webovych strankach 
        \item no evidence of fraud 
        \item posudek od OBSE , jen ze mame zrusit trest za pomluvu :)))
    \end{itemize}

    Americke prezidentske volby: 
    \begin{itemize}
        \item decentralizace volebniho system, dela si to kazdy stat sam, tezke kdyz chci konkretni data 
        \item no evidence of large scale fraud 
        \item posudek od OBSE
    \end{itemize}

    Beloruske prezidentske volby: 
    \begin{itemize}
        \item netransparentni
        \item multiple evidences of fraud 
        \item Voice, Zubr, Honest People dobra peace   
        \item vsichni vedi, ze v Belorusku si prezidenta proste nezvolis 
    \end{itemize}

    Limity digital detection of fraud 

    Other types of fraud

    Rozsireni (correlation among variables a tak) 

\end{koment}